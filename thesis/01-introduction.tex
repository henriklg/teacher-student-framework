\documentclass[thesis.tex]{subfiles}

\begin{document}

% ----------------------------------------------------------
\chapter{Introduction} \label{introduction}
% ----------------------------------------------------------
% Explain the aims and rationale for the physics case and what you have done. At the end of the introduction you should give a brief summary of the structure of the report. Motivate the reader and give overarching ideas. Describe what has been done and the structure of the report (how is it organized).
% ----------------------------------------------------------

% ----------------------------------------------------------
\section{Background and Motivation} \label{background_and_motivation}
% ----------------------------------------------------------
In this project we aim to design and develop a system for analyzing medical videos from a camera pill, as seen in Figure \ref{fig:pill-cam}. The pill is swallowed and records video of the entire digestive system The goal is to be able to detect different irregularities in the patients digestive system, like a colon polyp, Chron's disease, Colorectal cancer, etc. by using video object tracking, object detection, machine learning or other relevant tools.

Neural networks models that we would like to explore further for this purpose are Convolutional neural networks (CNN), Recurrent neural networks (RNN), Capsule neural networks, Long Short-Term memory networks and more.

The main idea is to go beyond image-based methods and also exploit the time factor of the data. 
The videos we will be using for this is delivered by Bærum Hospital, and is carefully labeled by using tools such as described in the paper \citetitle*{ExpertDriven15}. In this paper \citeauthor*{ExpertDriven15} presents a semi-supervised method to gather the annotations in a easy and time saving way \cite{ExpertDriven15}.

\begin{figure}[H] % fig:pill-cam
  \begin{center}
    \includegraphics[width=0.9\linewidth]{pill-cam.jpg}
    \caption[Illustration of how such a camera pill could look like]{Illustration of how such a camera pill could look like \cite{PillCamCamera}.}
    \label{fig:pill-cam}
  \end{center}
\end{figure}



% ----------------------------------------------------------
\section{Problem statement} \label{problem_statment}
% ----------------------------------------------------------
Colorectal cancer (CRC) is the third most common cause of cancer mortality for both men and women \cite{CancerStatistics10}, and it is a condition where early detection is of clear value for the ultimate survival of the patient. As statistics show that 15\% of male and female above 50 years are at risk, the procedure is recommended on a regular basis (every 3-5 years) for the population over 50, and from an earlier age for high-risk groups. 

Colonoscopy is a demanding procedure requiring an significant amount of time by specialized physicians, in addition to the discomfort and risks inherent in the procedure. Traditional methods based on colonoscopy are not cost-effective for population-based screening purposes, so only about 2-3\% of the target population is reached at present. 

The cost of a population screening program is prohibitively expensive. Colonoscopy is the most expensive cancer screening process in the US, with annual costs of \$10 billion dollars (\$1100 per person). In Norway we have similar costs of around \$1000 per person, with a time consumption of about 1 doctor-hour and 2 nurse-hours per examination. 

By researching an automatic system for a camera pill the aim is to greatly increase the number of patients that can be examined, i.e., making the public health care system more scalable and cost effective, while at the same time reducing the need for intrusive procedures like "bottom-up" examinations like colonoscopy.


% ----------------------------------------------------------
\section{Scope and limitations} \label{scope_and_limitations}
% ----------------------------------------------------------
%Limitations tell you what you won’t do due to project constraints that form boundaries, while Scope tells you what will be done and gives a total overhead view of a doable project within a period of time..

% limitations: 
%- biased dataset coming from one source (hospital) and one sensor (pillcam). AKA low diversity in sample size
%- collecting data/segmenting findings done by untrained students with some guidance
%- time constraints
%- small sample size (few medical videos)
%- searching for a given set of diseases

% scope: 
%- looking at only some AI models
%- compare atomatic classification systems against eachother
% ----------------------------------------------------------
Based on the described problem statement the scope of this thesis is to compare some prior selected automatic systems, with a focus on machine learning, for classification tasks in the medical domain. We will test the systems on video and still images taken from a wireless capsule endoscopy system, provided by Bærum Hospital. This data is generated with PillCam$^{TM}$ SB 3 System \footnote{\url{https://www.medtronic.com/covidien/en-us/products/capsule-endoscopy/pillcam-sb-3-system.html}} and gives us a wide view of the GI-tract. A limitation for our research will therefore be that our dataset lack a great deal of diversity. While we might have a wide range of 'patient' all the data contributions stem from one provider in one location. This might not be much of an issue due to GI-tract being quite homegenous over large populations. 

By using a tool provided by Augere Medical, we have sequencially classified each frame in some 45 PillCam videos. Although being assisted by doctor Thomas de Lange, with many years of experience, the dataset might not be perfectly labeled due to time constraints. But to the best of our efforts we managed to divide the data into 10 classes which we later will train our networks on.

Considering the scope of this thesis, we will limit ourselves to use some of the more common deep convolutional neural networks and a more novel stacked hourglass network. There are many other more common network used in the field of object classification like NN, RNN, stastisctial methods, but these methods are widely tested and we wish to push the envelope of machine learning systems on medical videos and also have a wide variety of ways to score our model against previously tested algorithms.


% ----------------------------------------------------------
\section{Research methods} \label{research_methods}
% ----------------------------------------------------------
Reference symbols: $\gls{x}$

% ----------------------------------------------------------
\section{Main Contributions} \label{main_contributions}
% ----------------------------------------------------------
% what did your research contribute to the field?


% ----------------------------------------------------------
\section{Thesis Outline} \label{thesis_outline}
% ----------------------------------------------------------
% the layout of the thesis
% ----------------------------------------------------------
This thesis is split into five chapter. Chapter one and two are mostly to introduce the reader to the topic and to fill in the necessary knowledge to understand the rest of the thesis. Thesis is structured as follows:

\begin{itemize}
\item In Chapter \ref{background} we discuss the literature that focus on the topic of automated lesion detection in computer systems.

\item In Chapter \ref{methodology} we present the details of design, implementation of system and the processing and collection of data.

\item In Chapter \ref{experiments} we present the experiments we have conducted and..

\item In Chapter \ref{conclusion} we provide a comprehensive overview of the results found and discus what that contributes to the field and propose some further work.

\end{itemize}


\end{document}
