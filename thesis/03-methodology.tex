\documentclass[thesis.tex]{subfiles}

\begin{document}

% ----------------------------------------------------------
\chapter{Methodology} \label{chap:methodology}
% ----------------------------------------------------------
% Present the method and system. Basically write everything you've done.
% ----------------------------------------------------------





% ----------------------------------------------------------
\section{Data collection} \label{sec:data_collection}
% ----------------------------------------------------------
% Present how you gathered the data and labels
% ----------------------------------------------------------
%TODO : add list of diseases from datasets (like stated in \ref{sec:colon_cancer})



\subsection{Kvasir-PillCam} \label{sec:kvasir_pillcam}
The dataset we used in our experiments consist of endoscopic videos collected from Bærum Hospital. Unlike Kvasir and Hyper-Kvasir datasets we have made the Kvasir PillCam dataset for the purpose of this thesis. In total we have 44 videos which have gone through some re encoding to reduce the file sizes, and also because the original encoding is proprietary Sony technology. After that the videos are uploaded to Augere Medical \footnote{\url{https://augere.md/}} tagging tool. The data export from Bærum also contains some findings for each video (if there is any) and are extracted, converted to frame number and that helped us a great deal with tagging the videos. We also have Thomas de Lange to thank, because he helped us a lot with the medical aspect of the classification process. When all 44 videos have been precisely labeled the dataset is exported from Augeres tagging tool and split into folders for each class. The folders/classes are given in table \ref{table:kvasir_pillcam}. In total we have 44 000 labeled images in 8 classes. The sample distribution across the eight classes is skewed depending on how many findings there are in the videos. Some findings occur often and some very rarely. The dataset also contain one class for 'normal' images, which there is quite a bit more of than findings. 

Imbalanced dataset pose a challenge for predictive algorithms as most learning algorithms are based on the assumption of an equal number of samples for each class. This results in models that have poor predictive performance, especially for minority class or classes. This is a great problem because in many medical datasets the minority class is the most important and therefore more sensitive for classification errors.

In addition to labeling the images the dataset also contain a JSON format file which stores coordinates for where in the frame the finding is located. The Kvasir Pillcam dataset will be an open-source dataset available for others scientists, and will later be grown to include more PillCam videos, both labeled and unlabeled samples.

\begin{table}
  \centering
  \begin{tabular}{ |c|c| }
  	\hline
  	Class number & Class name \\
    \hline
    0 & normal \\ 
    1 & polyp \\ 
    2 & polyrus \\ 
    \hline
  \end{tabular}
  \caption{PillCam class names and corresponding class numbers.}
  \label{table:kvasir_pillcam}
\end{table}






% ----------------------------------------------------------
\section{Data process} \label{sec:data_pipeline}
% ----------------------------------------------------------
% Present everything related to working with data files
% ----------------------------------------------------------

\subsection{Data preprocessing}
% data normalization, splitting, augmentation etc

\subsection{Data pipeline}
% tensorflow.data.Dataset pipeline with prefech, batching, shuffling, caching etc






% ----------------------------------------------------------
\section{System implementation} \label{sec:system_implementation}
% ----------------------------------------------------------
% Present the system architecture, some results perhaps
% ----------------------------------------------------------







% ----------------------------------------------------------
\section{Summary} \label{sec:C3-summary}
% ----------------------------------------------------------
% Present a summary of the chapter
% ----------------------------------------------------------


\end{document}